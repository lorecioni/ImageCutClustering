\documentclass[a4paper,12pt]{article} 
\usepackage[T1]{fontenc} 
\usepackage{enumerate}
\usepackage{graphicx}
\usepackage{listings}
\usepackage[a4paper,top=2.5cm,bottom=2.5cm,left=2.5cm,right=2.5cm]{geometry}


\title{\bf Features extraction and image clustering}
\date {19 January 2015}
\author{Lorenzo Cioni, Francesco Santoni\\\textit{{\small lore.cioni@gmail.com, francesco.santoni@gmail.com}}}

\begin{document}
\maketitle

\begin{abstract}
In this paper we discuss a method for extracting features from images containing handwritten US states to cluster them.
\end{abstract}

\tableofcontents

\section{Introduction}
Segmentation and clustering of big data is one of the main purpose of Artificial Intelligence at now. 
The aim of our work is to improve the performance of the clustering method implementend prevoiusly in an algorithm.

\section{The project}

\subsection{Troubles}

\section{Clustering}



\subsection{Features}

\subsubsection{Whitespace}  

\subsubsection{Loop}

\subsubsection{Dot}

\subsubsection{Diagonal line}

\subsubsection{Horizontal line}

\subsubsection{Vertical Line}

\subsubsection{Cross}  
  
\subsection{LCS} 

In order to compare the generated strings each others we use the \emph{Longest Common Subsequence} (LCS) algorithm for evaluating distances.

 
 
\subsection{Affinity Propagation}

\section{Results}

\section{Conclusion} 


\end{document}