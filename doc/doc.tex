\documentclass[a4paper,12pt]{article} 
\usepackage[T1]{fontenc} 
\usepackage{enumerate}
\usepackage{enumitem}
\usepackage{graphicx}
\usepackage{listings}
\usepackage{wrapfig}
\usepackage{subfigure}
\usepackage[a4paper,top=2.5cm,bottom=2.5cm,left=2.5cm,right=2.5cm]{geometry}
\usepackage{array}
\usepackage{tablefootnote}
\usepackage{float}
\setlength\extrarowheight{2pt}

\title{\bf Clustering of handwritten words based on structural features extraction}
\date {17 February 2015}
\author{Lorenzo Cioni, Francesco Santoni\\\textit{{\small lore.cioni@gmail.com, fsanto92@hotmail.it}}}

\begin{document}
\maketitle

\begin{abstract}

In this report we discuss our application for the extraction of primitive features from images of handwritten words and the generation of clusters of similar elements. In this case the words, compared to the names of American States and other countries, are extracted from forms of the 1930' U.S. census.
\end{abstract}

\tableofcontents

\section{Introduction}
Segmentation and clustering of large amounts of data is one of the main research fields of modern artificial intelligence.

The basis of our work is the collection of U.S. Census data in the year 1940 and fits into the process of digitalization of handwritten documents that characterizes our time. In this case we have scans of the census register and the main goal is to divide enrollees by State.

Document segmentation and extraction of the word corresponding to the State was developed previously by two of our colleagues in the course of \emph{Technology of Databases} and is the basis from which our work started.
The problem on which we have worked is the extraction of features from handwritten characters with the goal of creating clusters of similar words in order to facilitate the recognition by a human agent.

\begin{figure}[!ht]
\centering
\includegraphics[width=0.6\textwidth]{images/img1.jpg}
\caption{An example of a table containing census data}
\end{figure}
\section{The project}

The page that contains the data in digital form, contains a wealth of information about each person surveyed including name, gender, membership status and some secondary features such as work or the breed. 

All these data are housed in a moulded grid composed of numbered rows and columns for each field to conduct a census. In addition to the data of persons are reported on each archive also additional data related to censor or data relating to samples of the population.

The objective of the project was concerned with finding a particular grid area, the State of each person, from which extract handwritten text associated with the correct name of the State.

Following the data extraction, the project plans to group visually similar elements to form a collection of homogenous States. So each set and then will contain grid theoretically corresponding cutouts of the same State.

\subsection{Process steps}

We can summarize the project in three basic steps:
\begin{enumerate}
\item Search for the text area containing the words of interest
\item Row segmentation
\item Postprocessing of each cut image
\end{enumerate}

The last step, the clustering of images, is our point of interest: it has been rewritten and improved using new features extraction method and a new method for evaluating similarity and will be discussed in detail later.

\subsubsection{Word searching}

In this first phase the aim is to identify the region of the grid within which we find the State words. 

First of all we remove the black border of the document resulting from scanning. Then, using the technique of vertical histogram, we can locate the grid columns and, knowing the correct column's offset relative to the beginning of the document, just the one concerned is extracted is extracted.

\subsubsection{Row segmentation}

After the extraction of the column concerned from the document, we proceed with row segmentation.

Similar to the previous step, but using the histogram horizontal this time, we are able to determine with some accuracy the rows in the grid. In this case, however, was necessary to center the word in the extracted image: this is done by correcting the height of each row by the analysis of black spikes on the histogram.

At the end of this phase, each word is contained in a single image.

\subsubsection{Postprocessing}

In this postprocessing phase the individual images extracted are reworked in order to remove any  vertical or horizontal residual lines from the previous cut.

Delete row/column strokes can be about as bring back the black pixels related to the residual value of pure white (255 as grayscale).

This allows us to extract the most significant features in the next steps of processing.

\begin{figure}[!ht]
\centering
\vspace{0.3cm}
\includegraphics[width=0.5\textwidth]{images/img2.jpg}
\caption{An image extracted from document.}
\label{fig:extracted_image}
\end{figure}

As you can see in Figure \ref{fig:extracted_image}, the word is centered and the extra line have been removed (there is a line set to white). For each picture we proceed with feature extraction for handwritten characters and clustering. 

\subsection{Troubles}

The identification of words and their segmentation and has some strong issues.

A first problem is represented by the document skew. In this case, this may be due to a not perfectly horizontal scan of the original document. The presence of skew reduces the performance in searching for rows and columns of the grid, thus making impossible segmentation. In some cases, for example, is interpreted incorrectly the first column and this causes the extraction of the wrong column in the first stage of the process.

The main problems of the second phase are mainly related to the identification of extraneous lines. Because of the large number of dashed lines present isn't always possible to \emph{clean} the images and this can cause errors in clustering.

\begin{figure}[!ht]
 \centering
 \subfigure[An image with a dashed line]
   {\includegraphics[width=0.3\textwidth]{images/img3.jpg}}
 \hspace{5mm}
 \subfigure[The line intersects the word]
   {\includegraphics[width=0.3\textwidth]{images/img4.jpg}}
 \caption{Segmentation and postprocessing issues}
 \end{figure}

In other cases the word intersects directly gridlines. Because of this we may experience a loss of information about words by removing the line using the method described above.


\subsection{Features}

% % pezzo su come sono fatte stringhe


In this work we want to find \emph{primitive} features, resembling the possible types of strokes used to write a word, to characterize the sample from which they are extracted in order to perform clustering on their set.

The features are identified by the study of a sub-portion, or window, of the images extracted by the paper on which this work is based.

The features are located through a study of an area, the window, to which they are implicitly associated: as such there is no clear order of which of the features found in the area comes first.
Due to the absence of such \emph{native} order the string that defines a window is maintained consistent with the other strings trough the convention of generating the string with the features' identifiers always taking the same order, if present.

The chosen way to resolve the issue however presents the problem of making sliding windows inapplicable: this means that a sample must necessarily be cut in separate windows, which are allowed to overlap, with the corresponding effect that due to the random cut some features like \textit{loops} may not be recognised in an instance and recognised in another. 

\paragraph{Windows}

Currently we have chosen to cut the samples in windows of 32 pixels each with the exception of the last part of the sample that, deemed irrelevant to the end of characterization due to containing almost always white space, is simply ignored.

These windows are spaced only 16 pixels from the preceding and succeeding one, with the intent of creating overlap and lengthening the string that characterizes the samples.

While the simplest way to create the windows is simply creating a new PIX with the required coordinates it requires unnecessary passages and time in the creation of new objects. We have thus preferred to increase the complexity of the functions that search for the features, to whom we pass as a variable the original sample PIX with information about the \textit{offset} at which to start the search and the \textit{width} of the window.

% %parte su diversa width diverse feature

\subsubsection{Whitespace}  

The first feature that is search in the windows is the white space. This way if a window is identified as blank will not be necessary to proceed with the investigation of other features, saving time.

A window is identified as blank if the average pixel values is below a preset white threshold.

\subsubsection{Loop}

The feature that represents a loop is recognized only with maximum window size.
To locate a loop is initially found a black pixel. The idea behind is to imagine that we had found a point on the border and meet two transitions going in the same direction, first from black to white then from white to black. Between the two transitions there must be a minimum step above a preset threshold where the pixels are white.

If the above condition occurs then we verify that it is real in the same way along the horizontal axis. We then sit in the center of the loop (the median value of the segment described above) and check that, moving both right or left you get a transition from white to black after a suitable number of white pixels.

The implemented method has numerous problems related mainly to the identification of the center of the loop in the segment. With a value of threshold too high also might have to discard some loops too small. The method also does not take into account the thickness of the stroke.
However, with appropriate threshold values you can get good results and to identify obvious loops of various segments.

\subsubsection{Dot}

To find the points in the segment we proceed as in the previous case, seeking a first black pixel.

At that point, for it to be a Dot, there must be a circle of a given radius (whose value is preset and adjustable) in which the pixels are black and outside of that, for some white space.

Even in this case you have the problem of determining a good value of the radius of the circle, having to remember even the size of the stroke.

\subsubsection{Diagonal line}

The feature representing a diagonal line, both upward and downward facing, is extracted through a simple exhaustive scansion of the window for lines of connected black pixels that have an incline in a range of values and are sufficiently long.

At the moment the function doesn't account for the width of the lines found, thus diagonal features can be recognised in a shapeless blob of black pixel that is sufficiently big. 
   
There's a distinction for diagonal features that appear in the lower and upper bottom of the window.
At most in a window the function identifies a couple of upward and a couple of downward facing lines (lower and upper parts), once one has been found it stops searching for the same type.

\subsubsection{Cross}
The function, having found at most a lower and upper case for upward and downward diagonal lines, proceeds to confront the edge points of these lines: if they possibly intersects it extracts a \emph{crossing} feature with distinction if the crossing happens in the lower or upper part of the window. 

The problems with this sub-feature are that intersections of bottom and upper lines of the same type (e.g. both upward facing) with different inclines are not recognised, in the same way there's no consideration for intersections made from lines that are not the "primary" bottom and upper diagonals: if in a single windows are present more lower upwards diagonals and one of them other than the first intersects the recognised downward diagonal, such a crossing is ignored.

The crossing feature may also not necessarily be a complete intersection, in fact for recognition there just needs to be a merging of a downward and upward line. 


\subsubsection{Horizontal and Vertical line}
Both horizontal and vertical lines' features are extracted through
a simple scan of the window.
The horizontal lines requires a double-window for their implicit characteristic.
The function identifies a line if it finds a connected row or column of black pixels that has sufficient length, in particular it distinguishes the lines found in normal or long through an ulterior threshold.
Once a fitting line has been found the function keeps searching for more, continuing the scansion of the window after having moved a certain distance from the last line found in order not to confuse a particularly thick stroke as different separate lines.
There still persists the problem that the stroke width is not fully considered so a big blob of black pixels is seen as a series of horizontal and vertical lines, at the same time such an occurrence is rare so the presence of many horizontal and vertical lines ends up distinguishing the word in itself.     

\section{Evaluating distances}
After the extraction of good features from the words our aim is to construct a similarity matrix between different samples so that they can be used as input to the clustering algorithm. To do that we generate a measure of distance between couples of words through the use of the \emph{Longest Common Subsequence} algorithm in which the handled strings  are constructed through the appending of conventional identifiers associated with the features found in the word in a consistent order. 


\subsection{Longest Common Subsequence} 

In order to compare the generated strings one another we must define a distance on the samples.
The distance used in our work is based on the \emph{Longest Common Subsequence} (\textbf{LCS}) algorithm. 

The LCS algorithm has the aim of extracting from a set of sequences (in this case only two) the longest common subsequence, that is a sequence that is obtainable from both the starting sequences by deleting some elements without changing the order of the remaining ones.

LCS is a particular case of the \emph{Edit Distance} algorithm where the only allowed operations are insertion and deletion.
The distance associated with LCS in our current work is the number of insertion and deletions that must be applied to obtain the longest subsequence, in accordance with the Edit distance where the distance is calculated with the number of operations needed to morph a string in the other (in Edit Distance it's possible moreover to confer customizable costs to the substitutions).   

While the LCS algorithm may require high costs when applied concurrently to high numbers of sequences, in our case there exists an easy and light implementation that exploits \textit{dynamic programming}, in this type of implementation the cost ends up being $O(n*m)$ where $n$ and $m$ are the length of the compared strings.  


$$\label{LCS}
LCS(X_i,Y_j) =
 \left\{\begin{array}{ll}
 \displaystyle
 0 & if~i=0~ or~ j=0\\
 LCS(X_{i-1},Y_{j-1})\cup x_i & if~ x_i=y_j\\
 longest(LCS(X_i,Y_{j-1}), LCS(X_{i-1},Y_j)) & if~ x_i\neq y_j
 \end{array}\right.
$$


Obtained the Longest Common Subsequence between two strings the distance between them is thus:
$$ x.length() + y.length() - 2*LcsLength$$

where $x$ and $y$ are the strings and the $length()$ function returns the length of a string.

\subsection{Euclidean Distance} 

The Longest Common Subsequence is only used with strings. In order to improve the correctness of the words' distance matrix we combine LCS distance with the Euclidean Distance obtained through the use of the \textit{dimensional features}.

The Euclidean Distance between two samples is evaluated through the formula:
$$d_{a,b} = \sum_{i = 1}^{W} \sqrt{((t_{a}^{(i)} - b_{a}^{(i)}) - (t_{b}^{(i)} - b_{b}^{(i)}))^2 - (n_{a}^{(i)} - n_{b}^{(i)})^2}$$

where:
\begin{itemize}
\item $W$ is the word image width.
\item $t_{a}^{(i)}$ is the top black pixel in $i$ column in word $a$.
\item $t_{b}^{(i)}$ is the top black pixel in $i$ column in in word $b$.
\item $b_{a}^{(i)}$ is the bottom black pixel in $i$ column in in word $a$.
\item $b_{b}^{(i)}$ is the bottom black pixel in $i$ column in in word $b$.
\item $n_{a}^{(i)}$ is the number of transitions in $i$ column in in word $a$.
\item $n_{b}^{(i)}$ is the number of transitions in $i$ column in in word $a$.
\end{itemize}

\section{Clustering}
\label{Nostro_prog}
The clustering phase consists in the categorization of the various segments in homogeneous groups, so that, at the end of the process, each cluster contains only words corresponding to the same state.

At this stage the main goal is the extraction of good features from the words to be used for comparison. The aim is to construct a similarity matrix between different words so that they can be used as input to the clustering algorithm. To do that we generate a measure of distance between couples of words through the use of the \emph{Longest Common Subsequence} algorithm in which the handled strings  are constructed through the appending of conventional identifiers associated with the features found in the word in a consistent order. 

Unable to establish \emph{a priori} the optimal number of desired clusters makes the use of the \emph{Affinity Propagation} algorithm a necessity.

\subsection{Affinity Propagation}

Affinity Propagation is a clustering algorithm that identifies a set of \textit{exemplars} that represents the dataset\footnote{Brendan J. Frey, Delbert Dueck, \emph{Clustering by Passing Messages
Between Data Points}, http://www.sciencemag.org/, 2007}. The input of Affinity Propagation is the pair-wise similarities between each pair of data points\footnote{$s[i,j]$ for each data point is called preference and impacts the number of clusters.}, $s[i, j]  \forall  i, j = 1, \ldots, n$. Any type of similarities is acceptable thus Affinity Propagation is widely applicable.

Given similarity matrix $s[i, j]$, Affinity Propagation attempts to find the exemplars that maximize the net similarity, i.e. the overall sum of similarities between all exemplars and their member data points. The process of Affinity Propagation can be viewed as a message passing process with two
kinds of messages exchanged among data points: \emph{responsibility} and \emph{availability}. 

Responsibility, $r[i, j]$, is a message from data point $i$ to $j$ that reflects the accumulated evidence for how well-suited data point j is to serve as the exemplar for data point i. 

Availability, $a[i, j]$, is a message from data point $j$ to $i$ that reflects the accumulated evidence for how appropriate it would be for data point $i$ to choose data point $j$ as
its exemplar. All responsibilities and availabilities are set to $0$ initially, and their values are iteratively updated as follows to compute convergence values:

$$r[i, j] = (1-\lambda)\rho [i, j] + \lambda r[i, j]$$
$$a[i, j] = (1 -\lambda) \alpha[i, j] + \lambda a[i, j]$$

where $\lambda$ is a damping factor introduced to avoid numerical oscillations, and $\rho[i, j]$ and $\alpha[i, j]$ are, we call, \emph{propagating responsibility} and \emph{propagating availability}, respectively. 

$\rho[i, j]$ and $\alpha[i, j]$ are computed by the following equations:

$$
\rho[i, j] =
\left\{
\begin{array}{lr}
s[i, j] − \max_{k=j}a[i, k] + s[i, k] & i \neq j\\
s[i, j] − \max_{k=j}s[i, k]  &  i= j 
\end{array}
\right.
$$
$$
\alpha[i, j] =
\left\{
\begin{array}{lr}
min{0, r[j, j] + \sum_{k \neq i, j} \max 0, r[k, j]} & i \neq j\\
\sum_{k\neq j} \max {0, r[k, j]}  &  i= j 
\end{array}
\right.
$$

That is, messages between data points are computed from the corresponding propagating messages. The exemplar of data point $i$ is finally defined as:

$$arg \max {r[i, j] + a[i, j] : \forall \; j = 1, 2, \ldots,n}$$

\begin{figure}[H]
\centering
\includegraphics[width=0.73\textwidth]{images/ap.jpg}
\caption{How affinity propagation works}
\label{fig:ap}
\end{figure}

As described above, the original algorithm requires $O(n^2 t)$ time to update massages, where $n$ and $t$ are the number of data points and the number of iterations, respectively.
This incurs excessive CPU time, especially when the number of data points is large\footnote{Yasuhiro Fujiwara, Go Irie, Tomoe Kitahara, \emph{Fast Algorithm for Affinity Propagation}, 2009}. The Figure \ref{fig:ap} shows how affinity propagation works\footnote{Brendan J. Frey, Delbert Dueck, \emph{op. cit.}, Figure 1, p. 974}.

The clustering process described is then applied to the array of similarity calculated previously on features extracted from each words. Once you have run the calculation of clusters, segments are organized into individual folders to provide a visual result of the proceedings just completed. Each group is identified by a segment that represents the centroid of the cluster, which is the element to which all others in the group are closer.
\section{Results}

Let's explore the test phase and see the results in detail. Initially we ran debug tests on our personal PCs in order to easily modify the code, these starting tests and the tests utilized to determine the right value for the many defined constants are not shown here. 

All the tests shown below were performed on a single more powerful machine that has enabled us to work with much larger data in less time. The machine used is composed of two Xeon processors for a total of 16 cores 2.80 Ghz and 48 Gb of RAM.

The tests were performed on a growing number of scans, each containing a maximum of 50 lines from which we extracted the words that represent the states. For each set of scans we present three possible distances calculation: only LCS, only L1 (Euclidean distance) and LCS and L1 combined together through the formula presented in Section 5.

\vspace{3mm}

\begin{itemize}
\item \textbf{Estimated words (E)}: the number of words estimated on the basis of the number of scans(50x).
\item \textbf{Extracted words (W)}: the number of words actually mined and processed, as well as the corresponding percentage.
\item \textbf{Number of clusters (C)}: the number of clusters created in process.
\item \textbf{Running time (T)}: the total execution time, in seconds.
\item \textbf{Average precision (AP)}: the accuracy of the results, the average accuracy of clusters.
$$Precision_{average} = \frac{\sum_i P_i}{N_c}$$
where $P_i$ is the precision of cluster $i$ and $N_c$ is the number of clusters. 
\item \textbf{Precision (P)}: the accuracy of the results, the average accuracy of individual clusters weighted with the number of words.
$$Precision = \frac{\sum_i P_i * n_i}{N_w}$$
where $P_i$ is the precision of cluster $i$, $n_i$ is the number of words in cluster $i$ and $N_w$ is the number of words. 
\end{itemize}

In the next table we are going to show the main results of the tests. 

\begin{table}[!htbp]
\centering
\footnotesize
\begin{tabular}{|l | c | c | c | c | c | c |} 
 \hline 
 & \textbf{E} &  \textbf{W} & \textbf{C} & \textbf{T} & \textbf{AP} & \textbf{P} \\ [0.5ex] 
 \hline\hline
%% label & estimated words & extracted words & clusters & time & precision %%
16 scans (L1) & 800 & 550 & 55 & 15.52 & 65.06 & 58.00\\ 
16 scans (LCS) & 800 & 550 & 71 & 44.39  & 76.56 & 60.00\\ 
16 scans (LCS and L1) & 800 & 550 & 66 & 47.41 & 74.37 & 62.91\\ \hline
32 scans (L1) & 1600 & 800 & 67 & 38.58 & 59.85 & 52.87\\ 
32 scans (LCS) & 1600 & 800 & 92 & 94.54 & 72.97 & 55.50\\ 
32 scans (LCS and L1) & 1600 & 800 & 86 & 112.90 & 70.19 & 56.25\\ \hline
75 scans (L1) & 3750 & 2350 & 173 & 221.03 & 69.03 & 63.65\\ 
75 scans (LCS) & 3750 & 2350 & 189 & 1169.72 & 75.09 & 67.49\\ 
75 scans (LCS and L1) & 3750 & 2350 & 199 & 1203.28 & 78.01 & 71.16\\ \hline
130 scans (L1) & 6500 & 5050 & 425 & 5444.43 & 76.66 & 71.12\\ 
130 scans (LCS) & 6500 & 5050 & 441 & 5629.71 & 82.11 & 74.41\\ 
130 scans (LCS and L1) & 6500 & 5050 & 463 & 6729.62 & 84.87 & 77.94\\ \hline
500 scans (L1) & 25000 & 18146 & 3957 & 85545.11 & 90.66 & 77.89\\ 
500 scans (LCS) & 25000 & 18146 & 3316 &  76324.63 & 91.02 & 81.22\\ 
500 scans (LCS and L1) & 25000 & 18146 & 3941 & 138309.85\tablefootnote{During this test, the machine used was concurrently executing other tasks creating a bottleneck in the allotted memory, in all similar tests the time was in the order of 90k seconds.} & 92.27 & 82.14\\ 
 \hline
\end{tabular}
\caption{Main results}
\label{table:1}
\end{table}

As we can see in Table \ref{table:1} the number of extracted words is much lower than the estimated number of words. This is mainly due to the fact that not all census tables are completely filled: in some cases there are only a few lines or the states column was purposefully left blank. In other cases the absence of the state samples is due to errors occurring in the segmentation phase due to a wrong interpretation of the rows or the columns.

As we can see in Figure \ref{fig:precision} the accuracy of the cluster grows with the amount of words extracted. This phenomenon is due to the fact that Affinity Propagation works best with a large number of available data: the greater the number of words, the greater the chances of finding words similar between them, and then combine them within a single cluster.


\begin{figure}[!htbp]
\centering
\includegraphics[width=0.7\textwidth]{images/precisione.png}
\caption{Clustering precision relative to the number of scans}
\label{fig:precision}
\end{figure}

In the next table we will present the results regarding the time needed for the calculations. The running time is divided mainly into three distinct categories corresponding to different phases of the process: time needed for the extraction of features, time needed for the creation of similarity matrix (calculation of distances) and time needed for the clustering.

\begin{itemize}
\item \textbf{Features extraction time}: the time, in seconds, required for features extraction from the words.
\item \textbf{Evaluating distances time}: the time, in seconds, required to generate the similarity matrix with the distances between all the words extracted.
\item \textbf{Clustering time}: the time, in seconds, required for the creation of clusters.
\item \textbf{Running time}: the total execution time, in seconds.
\end{itemize}

\begin{table}[!htbp]
\centering
\footnotesize
\begin{tabular}{|l | c | c | c | c |} 
 \hline 
 & \multicolumn{1}{p{2cm}|}{\centering\bfseries Features extraction\\time (s)}&  \multicolumn{1}{p{2cm}|}{\centering\bfseries Evaluating distances\\time (s)} & \multicolumn{1}{p{2cm}|}{\centering\bfseries Clustering\\ time (s)} & \multicolumn{1}{p{2cm}|}{\centering\bfseries Total (s)} \\ [0.5ex] 
 \hline\hline
%% label & features & distances & clusters & total %%
16 scans (L1) & 6.81 & 5.02 & 3.69 & 15.52\\ 
16 scans (LCS) & 6.84 & 34.40 & 3.15 & 44.39\\ 
16 scans (LCS and L1) & 6.84 & 37.32 & 2.85 & 47.41\\ \hline
32 scans (L1) & 11.54 & 7.22 & 19.82 & 38.58\\ 
32 scans (LCS) & 11.27 & 74.72 & 8.55 & 94.54\\ 
32 scans (LCS and L1) & 11.34 & 89.96 & 11.60 & 112.90\\ \hline
75 scans (L1) & 33.65 & 69.00 & 118.38 & 221.03\\ 
75 scans (LCS) & 35.13 & 1002.63 & 131.96 & 1169.72\\ 
75 scans (LCS and L1) & 34.70 & 1070.92 & 97.66 & 1203.28\\ \hline
130 scans (L1) & 70.80 & 313.70 & 5059.93 & 5444.43\\ 
130 scans (LCS) & 68.17 & 4353.85 & 1207.69 & 5629.71\\ 
130 scans (LCS and L1) & 73.10 & 5608.82 & 1047.70 & 6729.62\\ \hline
500 scans (L1) & 289.33 & 4482.02 & 80773.76 & 85545.11 \\ 
500 scans (LCS) & 323.66 & 57699.51 & 18301.46 & 76324.63\\ 
500 scans (LCS and L1) & 403.41 & 61424.57 & $76481.87^5$ & $138309.85^5$\\ 
 \hline
\end{tabular}
\caption{Running time}
\label{table:2}
\end{table}

As we can see in Table \ref{table:2} the complexity is mainly due to the construction phase of the similarity matrix, that is the calculation of distances. This high operative cost occurs especially in calculating the LCS distance due to the fact that the structural strings of our words are very long and the cost of the algorithm is $O(nm)$, with $n$ and $m$ the lengths of the two strings, cost that must be considered in each comparison between couples of samples.

We sought long structural strings to obtain a deeper characterization of the words (and therefore make more accurate clustering), but this necessarily increases the calculation time.

\begin{figure}[!htbp]
\centering
\includegraphics[width=0.7\textwidth]{images/esecuzione}
\caption{Running time relative to the number of scans}
\label{fig:time}
\end{figure}

A parameter to evaluate the goodness of our application relatively to the task can be the number of clusters that have the absolute precision, that is contain only similar elements that are properly classified(due to human error sometimes the words in the debug files suffer a faulty categorization).

In the following table are presented the number of clusters that respect the above property and the number of items that they contain. 

\begin{itemize}

\item \textbf{Correct clusters}: the number of clusters with absolute precision, therefore containing only words equal to each other.
\item \textbf{Correct words}: the number of words within the Correct clusters and their corresponding percentage respect to all the elements.
\item \textbf{Single clusters}: clusters containing only one word.
\end{itemize}

\begin{table}[!htbp]
\centering
\footnotesize
\begin{tabular}{|l | c | c | c | c |} 
 \hline 
 & \multicolumn{1}{p{2cm}|}{\centering\bfseries Correct\\ clusters}&  \multicolumn{1}{p{2cm}|}{\centering\bfseries Correct\\words} & \multicolumn{1}{p{2cm}|}{\centering\bfseries Single\\clusters} & \multicolumn{1}{p{2cm}|}{\centering\bfseries Average words\\for cluster} \\ [0.5ex] 
 \hline\hline
%% label & clusters & words & percentage %%
16 scans (L1) & 12 & 18 & 10 &  1.50\\ 
16 scans (LCS) & 33 & 44 & 29 &1.33\\ 
16 scans (LCS and L1) & 25 & 45 & 18 & 1.80\\ \hline
32 scans (L1) & 11 & 11 & 10 & 1.00\\ 
32 scans (LCS) & 39 & 48 & 36 &1.23\\ 
32 scans (LCS and L1) & 32 & 43 & 28 & 1.34\\ \hline
75 scans (L1) & 39 & 172 & 11 & 4.41\\ 
75 scans (LCS) & 78 & 265 & 43 & 3.40\\ 
75 scans (LCS and L1) & 77 & 316 & 43 & 4.10\\ \hline
130 scans (L1) & 138 & 651 & 23 & 4.72\\ 
130 scans (LCS) & 208 & 713 & 107 &  3.43\\ 
130 scans (LCS and L1) & 215 & 860 & 89 & 4.00\\ \hline
500 scans (L1) & 2854 & 7354 & 151 & 2.58\\ 
500 scans (LCS) & 2902 & 7423 & 163 & 2.56\\ 
500 scans (LCS and L1) & 3020 & 7616 & 145 & 2.52\\ 
 \hline
\end{tabular}
\caption{Correct clusters}
\label{table:3}
\end{table}

\section{Compiling and running notes}

This software was developed entirely in C++, and uses an open source library of image processing and analysis, \emph{Leptonica}\footnote{Leptonica, \emph{a pedagogically-oriented open source site containing software that is broadly useful for image processing and image analysis applications} -http://www.leptonica.org/}, version 1.70.

\emph{Leptonica} provides many functions for manipulating images pixel by pixel using a high-level approach. Thanks to this library for example, you can draw up a diagram of projections, crop images, or find the connected components in a portion of the image.

To compile the code, once included the library described above, it is necessary, in the case of version of \emph{GCC/G++} less than 4.7, compile with version 11 of C++ that introduces support for threads.

To run this programm use the following parameters:
\begin{itemize}
\item \textbf{-d} to specify the images directory.
\item \textbf{-t} to specify the number of threads. Default is 2.
\item \textbf{-\--lcs} to use only LCS distance for building similarity matrix.
\item \textbf{-\--l1} to use only Euclidean distance for building similarity matrix.
\end{itemize}
\section{Conclusion}

Based on the results obtained we can note that the accuracy of the cluster grows with the amount of words extracted. This phenomenon is due to the fact that Affinity Propagation works best with a large number of available data: the greater the number of words, the greater the chances of finding words similar between them, and then combine them within a single cluster.

We can see then that, with the increase of the processed words, correct clusters also increase. This result enables to divide with a good approximation the words between them, even bringing the cluster that contains the same words in post processing.

The other factor to consider is the weather: as shown by the graph above the time increases nearly exponentially as a function of the number of words you want to extract. This, as shown, is due at the time of calculation of the distances between the structural strings of word with LCS. Using another calculation method you might reduce the necessary computing time .

\begin{thebibliography}{1}

\bibitem{vecchioBib}
Alessio Melani, Moreno Niccolai, \emph{Segmentazione e Clustering di Stati del censimento americano del 1940}.\hskip 1em plus
  0.5em minus 0.4em\relax Database Technology course, University of Florence, 2014.

\bibitem{clustBib}
Brendan J. Frey, Delbert Dueck, \emph{Clustering by Passing Messages
Between Data Points}.\hskip 1em plus
  0.5em minus 0.4em\relax http://www.sciencemag.org/, 2007.


\end{thebibliography}



\end{document}