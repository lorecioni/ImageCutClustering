\documentclass[a4paper,12pt]{article} 
\usepackage[T1]{fontenc} 
\usepackage{enumerate}
\usepackage{graphicx}
\usepackage{listings}
\usepackage[a4paper,top=2.5cm,bottom=2.5cm,left=2.5cm,right=2.5cm]{geometry}


\title{\bf Features extraction and image clustering}
\date {19 January 2015}
\author{Lorenzo Cioni, Francesco Santoni\\\textit{{\small lore.cioni@gmail.com, francesco.santoni@gmail.com}}}

\begin{document}
\maketitle

\begin{abstract}
In this paper we discuss a method for extracting primitive features from images containing handwritten US states to cluster them.
\end{abstract}

\tableofcontents

\section{Introduction}
Segmentation and clustering of big data is one of the main purpose of Artificial Intelligence at now. 
The aim of our work is to improve the performance of the clustering method implemented previously in an algorithm.

\section{The project}

\subsection{Troubles}

\section{Clustering}



\subsection{Features}

% % pezzo su come sono fatte stringhe


In this work we want to find "primitive" features, resembling the possible types of strokes used to write a word, to characterize the sample from which they are extracted in order to perform clustering on their set.
The features are identified by the study of a sub-portion,or window, of the images extracted by the paper on which this work is based.
The features are located through a study of an area, the window, to which they are implicitly associated: as such there is no clear order of which of the features found in the area comes first.
Due to the absence of such "native" order the string that defines a window is maintained consistent with the other strings trough the convention of generating the string with the features' identifiers always taking the same order, if present.
The chosen way to resolve the issue however presents the problem of making sliding windows inapplicable: this means that a sample must necessarily be cut in separate windows, which are allowed to overlap, with the corresponding effect that due to the random cut some features like \textit{Loop}s may not be recognised in an instance and recognised in another. 

\paragraph{Windows}
Currently we have chosen to cut the samples in windows of 32 pixels each with the exception of the last part of the sample that, deemed irrelevant to the end of characterization due to containing almost always white space, is simply ignored.
These windows are spaced only 16 pixels from the preceding and succeeding one, with the intent of creating overlap and lengthening the string that characterizes the samples.
While the simplest way to create the windows is simply creating a new PIX with the required coordinates it requires unnecessary passages and time in the creation of new objects. We have thus preferred to increase the complexity of the functions that search for the features, to whom we pass as a variable the original sample PIX with information about the \textit{offset} at which to start the search and the \textit{width} of the window.

% %parte su diversa width diverse feature

\subsubsection{Whitespace}  

\subsubsection{Loop}

\subsubsection{Dot}

\subsubsection{Diagonal line}
The feature representing a diagonal line, both upward and downward facing, is extracted through a simple exhaustive scansion of the window for lines of connected black pixels that have an incline in a range of values and are sufficiently long.
At the moment the function doesn't account for the width of the lines found, thus diagonal features can be recognised in a shapeless blob of black pixel that is sufficiently big.    
There's a distinction for diagonal features that appear in the lower and upper bottom of the window.
At most in a window the function identifies a couple of upward and a couple of downward facing lines (lower and upper parts), once one has been found it stops searching for the same type.
Of these couples the function search %searchs
 then if they possibly intersects, in such a case a "crossing" feature is extracted too, with distinction if the crossing happens in the lower or upper part. 

\subsubsection{Horizontal and Vertical line}
Both horizontal and vertical lines' features are extracted through
a simple scan of the window.
The horizontal lines requires a double-window for their implicit characteristic.
The function identifies a line if it finds a connected row or column of black pixels that has sufficient length, in particular it distinguishes the lines found in normal or long through an ulterior threshold.
Once a fitting line has been found the function keeps searching for more, continuing the scansion of the window after having moved a certain distance from the last line found in order not to confuse a particularly thick stroke as different separate lines.
There still persists the problem that the stroke width is not fully considered so a big blob of black pixels is seen as a series of horizontal and vertical lines, at the same time such an occurrence is rare so the presence of many horizontal and vertical lines ends up distinguishing the word in itself.     

\subsubsection{Cross}  
  
\subsection{LCS} 

In order to compare the generated strings each others we use the \emph{Longest Common Subsequence} (LCS) algorithm for evaluating distances.

 
 
\subsection{Affinity Propagation}

\section{Results}

\section{Conclusion} 


\end{document}