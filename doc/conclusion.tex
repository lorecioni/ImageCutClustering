\section{Conclusion}
In this project we wrote an application that after accepting as input U.S. Census documents returns clusters containing words that corresponds to the same state of birth.
Starting from a pre-existing work that already located and segmented the samples corresponding to the citizens' state, we added a more thorough search for the targeted area, and with the intent of improving the application by providing an alternate method of calculating the distance matrix we developed a different kind of features that we feel are more closely associated to the way a word is written through different strokes.
Through the use of these \textit{primitive} features and the already defined dimensional features we are able to generate distance matrices based only on one of the features' kind or on both after a reasoned combination.
We utilize then the Affinity Propagation algorithm to obtain the desired clusters.

Based on the results obtained we can note that the accuracy of the cluster grows with the amount of words extracted. This phenomenon is due to the fact that Affinity Propagation works best with a large number of available data: the greater the number of words, the greater the chances of finding words similar between them, and then combine them within a single cluster. At the same time, with the increase of the processed words there's also an increase in the rate of correct clusters. 

We note that the time needed to complete the calculations increases quadratically with the number of elements. This, as shown in Table 4, mainly due to the time needed by LCS to calculate the distance between each pair of words. Using other calculation methods one might reduce the necessary computing time.

From the tests the new features clearly provide an improvement on the older dimensional features, this improvement can further be expanded through the use of the mixed distance matrix. The improvement is sharp when low numbers of scans are considered but decreases at higher numbers where the precision of the 3 methods tend to converge.

The application overall succeeds in our intent although with not incredible improvements on the preceding work, possible avenues of optimization are in the definition of the strings associated to the features, taking in consideration that an increase in the string length is directly proportional to an increase in LCS distance's calculation time.

Other possible and obvious improvement, although that  would require an overall rewriting of the project from it's foundation, would be the association of each segmented word with its coordinates in the general census image: this cannot be done at the moment because the coordinates of a segment are ignored since the preexisting inherited segmentation steps that simply cuts the words without any reference to their place in the general image.  
This improvement would confer an effective practical utility to the application; at the same time it would permit a betterment of the debug and testing phase through the possible use of the same coordinates in the search of the word corresponding to the segment, rather than having to rely on each segment's row number.
   